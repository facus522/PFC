%---------------------------------------------------------------------
%
%                      resumen.tex
%
%---------------------------------------------------------------------
%
% Contiene el cap�tulo del resumen.
%
% Se crea como un cap�tulo sin numeraci�n.
%
%---------------------------------------------------------------------

\chapter{Resumen}
\cabeceraEspecial{Resumen}

A lo largo de los a�os se han utilizado las encuestas como m�todo de recopilaci�n de datos sobre una determinada poblaci�n de inter�s, para luego poder llevar a cabo acciones sobre ella. Tanto la recopilaci�n de datos, como el monitoreo y la evaluaci�n son tareas que requieren de una amplia planificaci�n met�dica y de mucho tiempo.
\\

Espec�ficamente en el �mbito de la salud, la recolecci�n de datos en gran escala llega a tener un impacto masivo, ya que con ellos se puede llevar a cabo un seguimiento de la informaci�n y ayudar a tomar medidas inmediatas, tanto preventivas como informativas, sobre verdaderas prioridades, obteniendo como resultado la reducci�n de enfermedades, mejoras en la calidad de vida, etc.
\\

Actualmente, la Facultad de Ciencias M�dicas de la Universidad Nacional del Litoral se encarga de realizar encuestas sobre salud de la manera tradicional, es decir con "l�piz y papel". �sta metodolog�a no solamente hace que el proceso est� propenso a errores, sino que tambi�n aumenta la dificultad de realizaci�n en forma masiva. Lo cual genera un aumento en los costos de transacci�n y demora en el procesamiento de los datos, imposibilitando de �sta forma tomar medidas urgentes.
\\

Aprovechando los distintos beneficios que proporcionan las tecnolog�as hoy en d�a, la finalidad del presente proyecto es llevar a cabo el desarrollo de una aplicaci�n m�vil que digitalice las encuestas, llevando a una reducci�n de las labores mencionadas, con un procesamiento de datos y obtenci�n de resultados en cuesti�n de segundos. A su vez se brindar� la posibilidad de obtener los datos de forma geo-referenciada y exportarlos en formato Excel, para que distintas ramas de la medicina trabajen con ellos, as� como tambi�n los bio-estad�sticos que vuelcan dichos datos en distintos softwares de estad�stica.
\\

Con la finalidad de independizar los datos que representan el n�cleo del proyecto de la aplicaci�n, se llevar� a cabo el desarrollo de un Web Service el cual ser� consumido a la hora de cargar informaci�n y realizar los ABMs correspondientes.
\\

\smallskip
\noindent \textbf{Palabras claves:} Encuestas, Dispositivo M�vil, Aplicaci�n M�vil, Web Service, Geo-Referencias, Epidemiolog�a, GPS, Endpoint.

\newpage
\thispagestyle{empty}
\mbox{ }

\endinput
% Variable local para emacs, para  que encuentre el fichero maestro de
% compilaci�n y funcionen mejor algunas teclas r�pidas de AucTeX
%%%
%%% Local Variables:
%%% mode: latex
%%% TeX-master: "../Tesis.tex"
%%% End:
