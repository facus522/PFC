%---------------------------------------------------------------------
%
%                      agradecimientos.tex
%
%---------------------------------------------------------------------
%
% agradecimientos.tex
% Copyright 2009 Marco Antonio Gomez-Martin, Pedro Pablo Gomez-Martin
%
% This file belongs to the TeXiS manual, a LaTeX template for writting
% Thesis and other documents. The complete last TeXiS package can
% be obtained from http://gaia.fdi.ucm.es/projects/texis/
%
% Although the TeXiS template itself is distributed under the 
% conditions of the LaTeX Project Public License
% (http://www.latex-project.org/lppl.txt), the manual content
% uses the CC-BY-SA license that stays that you are free:
%
%    - to share & to copy, distribute and transmit the work
%    - to remix and to adapt the work
%
% under the following conditions:
%
%    - Attribution: you must attribute the work in the manner
%      specified by the author or licensor (but not in any way that
%      suggests that they endorse you or your use of the work).
%    - Share Alike: if you alter, transform, or build upon this
%      work, you may distribute the resulting work only under the
%      same, similar or a compatible license.
%
% The complete license is available in
% http://creativecommons.org/licenses/by-sa/3.0/legalcode
%
%---------------------------------------------------------------------
%
% Contiene la p�gina de agradecimientos.
%
% Se crea como un cap�tulo sin numeraci�n.
%
%---------------------------------------------------------------------

\chapter{Agradecimientos}

\cabeceraEspecial{Agradecimientos}

A mi familia que permiti� que hoy en d�a pueda estar cumpliendo este objetivo y sue�o personal. A mi padre Carlos que aunque ya no est� f�sicamente conmigo, fue el gran responsable junto a mi madre para que hoy en d�a sea lo que soy. Toda la educaci�n, valores y respeto transmitidos por ellos, tanto en lo acad�mico como en la vida misma, fue indispensable para cumplir esta meta. 
\\

En especial a mi madre Viviana que en los momentos m�s duros, llev� sola adelante la familia con gran determinaci�n y coraje, continuando con mi educaci�n y la de mis hermanos. Agradecerle por su apoyo incondicional y confianza, brind�ndome la posibilidad de dedicarme cien por ciento al estudio de la carrera, sin tener que salir a trabajar. 
\\

A cada una de las personas que componen la Facultad de Ingenier�a y Ciencias H�dricas. A todos aquellos profesores que he tenido a lo largo de la carrera, que en mayor o menor medida contribuyeron para mi formaci�n, tanto profesional como personal.
\\

Por �ltimo, quisiera dar las gracias a todos mis amigos que supieron estar conmigo durante esta etapa de mi vida en todos sus aspectos. Muchos de ellos de toda la vida, tantos otros adquiridos a lo largo de los a�os y durante mi trayecto en la universidad. En especial al grupo de ``Inform�ticos'' y sus ``Pe�as de martes'', los cuales fueron de suma importancia para poder llevar adelante la carrera a la par. Apoy�ndonos, solidariz�ndonos y compartiendo conocimientos entre todos, logrando que el objetivo sea m�s f�cil de cumplir.
\\

Me llevo hermosos recuerdos y an�cdotas que perdurar�n por siempre, pudiendo decir orgullosamente que estudi� en la Facultad de Ingenier�a y Ciencias H�dricas de la Universidad Nacional del Litoral.

\newpage
\thispagestyle{empty}
\mbox{ }

\endinput
% Variable local para emacs, para  que encuentre el fichero maestro de
% compilaci�n y funcionen mejor algunas teclas r�pidas de AucTeX
%%%
%%% Local Variables:
%%% mode: latex
%%% TeX-master: "../Tesis.tex"
%%% End:
