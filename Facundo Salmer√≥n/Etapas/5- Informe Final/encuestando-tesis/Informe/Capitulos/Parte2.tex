%---------------------------------------------------------------------
%
%                          Parte 2
%
%---------------------------------------------------------------------
%
% Parte2.tex
% Copyright 2009 Marco Antonio Gomez-Martin, Pedro Pablo Gomez-Martin
%
% This file belongs to the TeXiS manual, a LaTeX template for writting
% Thesis and other documents. The complete last TeXiS package can
% be obtained from http://gaia.fdi.ucm.es/projects/texis/
%
% Although the TeXiS template itself is distributed under the 
% conditions of the LaTeX Project Public License
% (http://www.latex-project.org/lppl.txt), the manual content
% uses the CC-BY-SA license that stays that you are free:
%
%    - to share & to copy, distribute and transmit the work
%    - to remix and to adapt the work
%
% under the following conditions:
%
%    - Attribution: you must attribute the work in the manner
%      specified by the author or licensor (but not in any way that
%      suggests that they endorse you or your use of the work).
%    - Share Alike: if you alter, transform, or build upon this
%      work, you may distribute the resulting work only under the
%      same, similar or a compatible license.
%
% The complete license is available in
% http://creativecommons.org/licenses/by-sa/3.0/legalcode
%
%---------------------------------------------------------------------

% Parte en donde se explica lo "copado" que es Learninspy

\partTitle{Encuestando FCM}

\partDesc{En esta segunda parte del documento se har� referencia a todo lo relativo al dise�o y desarrollo del Web Service en funci�n de los conceptos presentados en la parte inicial, incorporando conceptos m�s t�cnicos como ser la base de datos y los frameworks utilizados, entre otras cosas. Adem�s, se har� lo propio en el lado cliente especificando el proceso de dise�o y desarrollo de la aplicaci�n junto con las tecnolog�as utilizadas y una noci�n de los componentes utilizados. Ambos sistemas componen el producto final denominado \textit{Encuestando FCM} (Figura \ref{fig:encuestandoFCM}).
\\

\begin{figure}[h!]
	\begin{center}
		\includegraphics[width=0.8\textwidth]%
		{Imagenes/Bitmap/icono}
		\caption{�cono de la aplicaci�n desarrollada}
		\label{fig:encuestandoFCM}
	\end{center}
\end{figure}
}



\partBackText{}

\makepart
