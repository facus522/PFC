%---------------------------------------------------------------------
%
%                          Parte 1
%
%---------------------------------------------------------------------
%
% Parte1.tex
% Copyright 2009 Marco Antonio Gomez-Martin, Pedro Pablo Gomez-Martin
%
% This file belongs to the TeXiS manual, a LaTeX template for writting
% Thesis and other documents. The complete last TeXiS package can
% be obtained from http://gaia.fdi.ucm.es/projects/texis/
%
% Although the TeXiS template itself is distributed under the 
% conditions of the LaTeX Project Public License
% (http://www.latex-project.org/lppl.txt), the manual content
% uses the CC-BY-SA license that stays that you are free:
%
%    - to share & to copy, distribute and transmit the work
%    - to remix and to adapt the work
%
% under the following conditions:
%
%    - Attribution: you must attribute the work in the manner
%      specified by the author or licensor (but not in any way that
%      suggests that they endorse you or your use of the work).
%    - Share Alike: if you alter, transform, or build upon this
%      work, you may distribute the resulting work only under the
%      same, similar or a compatible license.
%
% The complete license is available in
% http://creativecommons.org/licenses/by-sa/3.0/legalcode
%
%---------------------------------------------------------------------

% Definici�n de la primera parte del manual

\partTitle{Conceptos b�sicos}

\partDesc{En esta primera parte del documento se establece un marco te�rico en funci�n de explicar e introducir conceptos claves que componen el presente proyecto. Esto comprende los elementos que componen la especificaci�n del proyecto y los fundamentos que sustentan las soluciones propuestas para alcanzar los objetivos planteados, junto con los alcances. Es importante que el lector conozca para la lectura posterior todos los conceptos presentados en �sta parte, debido a que se han aplicado cuestiones de los mismos y que devienen del an�lisis realizado en cuanto a la problem�tica y el enfoque propuesto para su soluci�n.}

\partBackText{}

\makepart
